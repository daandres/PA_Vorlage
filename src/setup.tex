%% Optional Packages. 
%% Choose packages you want by using comments in the list below. 

\usepackage[
%% en, %% English text and wording
de, %% German text and wording
%%singlepage, %% Put sitenumbers always on the right
doublepage, %% Put sitenumbers alternating on the left and on the right. The first number starts always right, so be sure that you insert a blank page with \shipout\null
acronym, %% Use acronyms and print a acronym table. Use acronym.tex to add acronyms
figures, %% make list of figures and include graphicx package
tables, %% make list of tables and include table packages, also supertabular
%% listings, %% Use listings for source code and create table of listings
intro, %% include introduction file and make bookmark
bib, %% make list of references at the end and import bib package. Use bib.bib as bibliography
appendix, %% includes appendix. add sections in the file appendix.tex
PA, %% inserts at the titlepage the company and such stuff
%%Study, %% insert at the titlepage information without company and such stuff
2010, %% Bis Jahrgang 2010 --> Ehrenwörtliche Erklärung
%% 2011, %% Ab Jahrgang 2011 --> Ehrenwörtliche Erklärung
]{optional} 
 
\opt{en}{\usepackage[american]{babel}} 
\opt{de}{\usepackage[ngerman]{babel}}

% %%% Folgende Angaben ausfüllen:

\newcommand{\student}{Firstname Lastname} 
\newcommand{\matrikel}{133742}  
\newcommand{\kurs}{TAI10ABC}  

\newcommand{\reporttitle}{Titel}  
\newcommand{\reportsubtitle}{Subtitel}

\opt{en}{\newcommand{\reportstart}{startdate}} 
\opt{de}{\newcommand{\reportstart}{startdatum}} 
\opt{en}{\newcommand{\reportend}{enddate}}
\opt{de}{\newcommand{\reportend}{enddatum}}
\opt{en}{\newcommand{\handoverdate}{31/13/3113}}
\opt{de}{\newcommand{\handoverdate}{31.13.3113}}
 
\newcommand{\tutor}{Name Tutor}
\newcommand{\companylogo}{icon-hinweis} %% filname of logo for titlepage
\newcommand{\companylogosmall}{} %% filename for logo on all other pages
\opt{Study}{
	\newcommand{\reportsemester}{5. - 6. Semester}
}
\opt{PA}{
	\newcommand{\businessunit}{Business Unit} 
	\newcommand{\department}{Department}
	\newcommand{\company}{Company}
	\newcommand{\companyshort}{COMP} %% used in Confidential flag on each site
	\opt{en}{\newcommand{\educationdepartment}{Education department}}
	\opt{de}{\newcommand{\educationdepartment}{Ausbildungsabteilung}}
	\opt{en}{\newcommand{\lokation}{Mannheim}}
	\opt{de}{\newcommand{\lokation}{Mannheim}}
	\newcommand{\manager}{Education Manager}
}



\newcommand{\dhbws}{DHBW Mannheim}
\opt{en}{\newcommand{\dhbw}{BW Cooperative State University Mannheim, Germany}}
\opt{de}{\newcommand{\dhbw}{Duale Hochschule Baden-W{\"u}rttemberg Mannheim}}
\opt{en}{\newcommand{\studiengang}{applied computer science}}
\opt{de}{\newcommand{\studiengang}{Angewandte Informatik}}
\newcommand{\prof}{Prof. Dr. H. Hofmann}

\opt{listings}{
	\newcommand{\listingssetup}{
		\lstset{
			language=Java,                  % the language of the code
			basicstyle=\footnotesize,       % the size of the fonts that are used for the code
			numbers=left,                   % where to put the line-numbers
			numberstyle=\tiny\color{gray},  % the style that is used for the line-numbers
			stepnumber=0,                   % the step between two line-numbers. If it's 1, each line 
			                                % will be numbered
			numbersep=5pt,                  % how far the line-numbers are from the code
			backgroundcolor=\color{white},  % choose the background color. You must add \usepackage{color}
			showspaces=false,               % show spaces adding particular underscores
			showstringspaces=false,         % underline spaces within strings
			showtabs=false,                 % show tabs within strings adding particular underscores
			frame=single,                   % adds a frame around the code
			rulecolor=\color{black},        % if not set, the frame-color may be changed on line-breaks within not-black text (e.g. commens (green here))
			tabsize=2,                      % sets default tabsize to 2 spaces
			captionpos=b,                   % sets the caption-position to bottom
			breaklines=true,                % sets automatic line breaking
			breakatwhitespace=false,        % sets if automatic breaks should only happen at whitespace
			title=\lstname,                 % show the filename of files included with \lstinputlisting;
			                                % also try caption instead of title
			keywordstyle=\color{blue},      % keyword style
			commentstyle=\color{dkgreen},   % comment style
			stringstyle=\color{gray},       % string literal style
			escapeinside={\%*}{*)},         % if you want to add LaTeX within your code
			morekeywords={*,...}            % if you want to add more keywords to the set
		}
	}
}

%%%%%%%%%%%%
%% Folgende Angaben müssen nicht verändert werden. Werden aber zur einfacheren ANpassung (falls nötig) hier aufgeführt
\opt{PA}{
	\opt{en}{\newcommand{\reporttype}{Internship Report}}
	\opt{de}{\newcommand{\reporttype}{Projektarbeit}}
}
\opt{Study}{
	\opt{en}{\newcommand{\reporttype}{Study Report}}
	\opt{de}{\newcommand{\reporttype}{Studienarbeit}}
}
