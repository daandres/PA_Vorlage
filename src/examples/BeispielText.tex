\section{Strukturierung und Gestaltung}
In Abschnitt \ref{leererAbschnitt} steht nichts, außer unter \ref{interessanterUnterabschnitt} auf Seite \pageref{interessanterUnterabschnitt}.


\subsection{Ein Unter-Abschnitt}
\label{leererAbschnitt}



\subsubsection{Ein Unter-Unter-Abschnitt}
\label{interessanterUnterabschnitt}

\paragraph{Ein Absatz mit Überschrift}
Mit den Standard-Einstellungen stehen für kurze Dokumente (des Typs "`article"') drei nummerierte Gliederungsebenen zur Verfügung.

\subparagraph{Ein Unter-Absatz mit Überschrift}
Weiter gibt es darunter zwei nicht nummerierte Gliederungsebenen.



\subsection{Der zweite Unter-Abschnitt}
Dient der Illustration und enthält keinen weiteren Inhalt,
\begin{tabular}{|l|c|r|}
\hline
eins & zwei & drei \\
\hline
Testeintrag & Testeintrag & Testeintrag \\
\hline
\end{tabular}
dafür eine Tabelle mitten im Text und ohne Bezeichnung.
Viel schöner ist Tabelle \ref{testTabelle}.

\subsection{Aufzählungen}
Diese gibt es mit nummerierten Labels:

% der optionale Parameter enthält die Art der Darstellung, 
% Alternativen sind z.B. a), i, (I),  etc.
\begin{enumerate}[$\rightarrow$ 1:]
\item Starten Sie am besten mit einem "`All-in-One"'-Installationspaket, wie z.B.\ jenem unter \url{miktex.org}.

\item Mehr Spaß macht es mit einem vernünftigen Editor wie etwa TeXniccenter (\url{texniccenter.org}).
\end{enumerate}

% durch {} kann man beliebige Abschnitte im Quellcode klammern
% das Kommando \small wirkt sich nur innerhalb des geklammerten
% Blocks aus

{\Large
oder als Aufzählung oder Nummerierung:
}
{\small
\begin{itemize}
\item Mac-Benutzern sei TexShop empfohlen. Unter \url{http://pages.uoregon.edu/koch/texshop/} gibt es mit TeX Live ein Paket aus Compiler und Editor.

\item JabRef~\cite{JabRefOnlineDoku} ist ein Werkzeug zur Literaturverwaltung. (Damit die Literatur-Verweise funktionieren, müssen Sie BibTeX aufrufen.)
\end{itemize}
}

\subsection{Gestaltung}
Oftmals wird eine 
% (emph steht für "emphasize")
\emph{Hervorhebung} 
einzelner Wörter benötigt, 
% (textbf steht für "text bold face")
\textbf{Fettschrift}
im Fließtext mag -- sparsam eingesetzt --
zuweilen sinnvoll sein.\footnote{Über Geschmack lässt sich ja streiten.}

% der optionale Parameter "h" gibt an, dass der Block
% mit der Abbildung vorzugsweise an der aktuellen Position,
% alternativ unten ("botton") platziert werden soll
\begin{table}[hb]
\begin{center}
\begin{tabular}{c||r|l}
  & \textbf{Software}	& \textbf{Kosten} \\  
  \hline
  1 & Word & 100 EUR, für Studis kostenlos \\
  \hline
  2 & \LaTeX & Open Source \\
  \hline
\end{tabular}
\end{center}
% Beschriftung festlegen:
\caption{Von Studierenden verwendete Software zur Erstellung eines Berichts.} 
% ein Label definieren, mit dessen Hilfe man (an beliebiger Stelle im Dokument) Bezug nehmen kann:
\label{testTabelle}
\end{table}

Für mathematische Formeln gibt es einen eigenen Modus, um etwa $\forall e \in
\mathcal{K} \: \: \exists d \in \mathcal{K} \: \: \forall m \in \mathcal{P}:
D_d(E_e(m)) = m$
% das $\:$ definiert übrigens einen horizontalen Abstand
oder $2 = 5 \tmod 3$ zu schreiben. (Ich verwende $a \tmod m$, wenn der Rest
gemeint ist und ``$\mod$'', wenn es rechts von einer Kongruenzgleichung wie \[
  2^{20}\equiv 2^{3\times6+2}\equiv (2^6)^3 2^2\equiv 2^2 \equiv 4 \mod 7
\] steht, die in $(\Z/7\Z)^*$ gilt.
% Grafiken werden in separaten Dateien gespeichert und lassen sich wie in
% Abbildung \ref{labelsSolltenMoeglichstAussagekraeftigGewaehltSein} gezeigt
% einbinden.
Für die Formatierung von Quellcode jeglicher Couleur gibt es z.B. das Paket
\emph{listings}, unter \url{ctan.org} finden Sie noch eine ganze Menge mehr
\ldots Umlaute sind auch kein Problem, wenn Sie \verb|umlaut.sty| einbinden, bei
``Anführungszeichen'' werden die öffnenden und schließenden unterschieden
(schauen Sie im Quelltext nach!).

% der optionale Parameter "t" gibt an, dass der Block
% mit der Abbildung vorzugsweise oben ("top") platziert werden soll
% \begin{figure}[t]
% \begin{center}
% \includegraphics[width=3cm]{grafiken/DHBW-Logo.png}
% % sorgt für etwas horizontalen Abstand
% \hspace{0.5cm}
% % das nächste Bild
% \includegraphics[width=2cm]{grafiken/testbild.pdf}
% \end{center}
% \caption{Das Logo der DHBW (als PNG), daneben ein anderes Bild (PDF). Beide Formate können zur Erzeugung eines PDF-Dokuments unkompliziert verwendet werden.}
% \label{labelsSolltenMoeglichstAussagekraeftigGewaehltSein}
% \end{figure}

\clearpage
\section{Links}

Falls Sie Folien mit \LaTeX\ machen möchten, schauen Sie mal hier\footnote{
\url{http://amath.colorado.edu/documentation/LaTeX/prosper/}} oder hier\footnote{ 
\url{http://www.physik.uni-freiburg.de/~tooleh/latex_beamerkurs.pdf}}.
Für den Anfang empfehle ich die Kombination mit Powerpoint (Formeln kopieren Sie als Grafik in die Präsentation hinein).
Online-\LaTeX-Editoren für einzelne Formeln gibt es hier\footnote{ 
\url{http://www.sciweavers.org/free-online-latex-equation-editor}}
oder hier\footnote{\url{http://www.codecogs.com/latex/eqneditor.php}}. Von unschätzbarem Wert ist zuweilen Detexify\footnote{\url{http://detexify.kirelabs.org/classify.html}}: Sie zeichnen ein Symbol von Hand und bekommen die entsprechenden \LaTeX-Schreibweise genannt.

